\chapter{Generating open educational resources for university-level bioinformatics courses}

Rapid advances in next-generation sequencing (NGS) technologies have improved accessibility for experimentalists to generate genomic data at scale, but the barrier to entry to learning the computational skills necessary to analyze these datasets remains high. Despite computational courses being slowly integrated into the classical undergraduate Biology curricula, the breadth of scientific and technical knowledge needed to succeed in bioinformatics courses renders them inaccessible to individuals with incomplete foundations.

For many bioinformatics graduate programs, there can be an expectation for trainees to already have a baseline knowledge of programming and bioinformatics pipeline development. Inevitably, there is usually a proportion of admitted students that are non-computational. Not addressing this knowledge gap amongst non-computational scientists contributes to issues with student retention and morale within the program, especially for students of minoritized backgrounds. To directly address this need, I made it my mission in graduate school to develop inclusive teaching strategies in academically diverse classrooms to provide students with the skills necessary to confidently perform and understand bioinformatics analysis. Additionally, I advocated for and succeeded in making expectations of incoming bioinformatics graduate students clearer to improve the retention of trainees. As a consequence of the quarantine in response to the global SARS-CoV-2 pandemic, I taught in-person and fully online modalities. 

%%%%%%%%%%%%%%%%%%%%%%%%%%%%%%%%%%%%%%%%%%%%%%%%%%%%%%%%%%%%%%%%%%%%%%%%%%%%%%%%
\section{Introduction}
%%%%%%%%%%%%%%%%%%%%%%%%%%%%%%%%%%%%%%%%%%%%%%%%%%%%%%%%%%%%%%%%%%%%%%%%%%%%%%%%

\subsection{Bioinformatics as a specialized data science discipline}

Massively parallel or next-generation sequencing (NGS) provides researchers with an exceedingly flexible set of techniques to study various types of biological sequence data at a large scale [CITE]. Currently, most sequencing is performed in research laboratories that need sustainable strategies for handling computational processing and data storage [CITE]. Ergo, it has become necessary for biological and biomedical scientists at all educational levels to have some basic computational education for successful research [CITE]. 

The computational field of data science is an interdisciplinary discipline that utilizes algorithms, statistics, and scientific methods to extrapolate knowledge from various data types [CITE]. Thus as an amalgamation of disciplines, bioinformatics can be considered a subset of data science as it requires the ability to integrate concepts across biology, mathematics, computer science, and statistics [CITE]. Additionally, bioinformatics requires substantial subfield-specific knowledge about particular computational tools and the biological context in which data was generated to generate accurate interpretations of data [CITE]. Therefore, universities should consider this necessary breadth of knowledge in designing new undergraduate Biology curricula for their students to apply computational skills appropriately. 

While universities have started integrating computational modules into undergraduate and graduate biology student training, these modifications have not happened consistently across programs. Additionally, integrating computational coursework into current programs does not address the learning gap for scientists that wish to learn bioinformatics later in their careers when they do not have access to a classroom [CITE]. These learners often seek out opportunities to take computational courses in their own time, including in the university setting where many work. Unfortunately, the rising interest in computer science has imposed course enrollment caps in introductory programming and algorithms undergraduate courses due to the unmatched supply of available classes and instructors [CITE]. These enrollment caps then severely limit the opportunities for non-undergraduate individuals to supplement their professional experience with basic computational skills in an academic setting [CITE]. An added issue is that for the few fortunate enough to enroll in a pure computer science course and learn the computational problem-solving mindset, these courses can then be difficult to translate directly into running bioinformatics pipelines [CITE]. Thus, there is a need to develop accessible university-level bioinformatics course material.

\subsection{Placing bioinformatics in the context of discipline-based education research}

Educational researchers focus on scientific investigation of topics within the field of education to improve teaching and learning practices [CITE]. While educational researchers can focus on general teaching topics, such as effective teaching methods for learners of various ages, discipline-based education research (DBER) evaluates learning and teaching in a particular discipline, such as biology or computer science [CITE]. 

While DBER looks at different disciplines separately, there are concrete similarities in their multidisciplinary nature and overall goal of improving the learning experience for students at the primary, secondary, and higher education levels within their respective fields. Computer science education or computing education research addresses learning and teaching in computer science [CITE]. For this DBER field, the Association for Computing Machinery runs a special interest group (SIG) on computer science education (CSE) research known as SIGCSE, whose affiliated conferences are some of the top venues for educational scholars to discuss topics related to computing and teaching methods [CITE]. Computing education research covers an array of questions, including studying the retention of students, the difficulties of novice programmers, and the effectiveness of learning tools employed in the classroom [CITE]. Similarly, biology education research concerns the promotion and accessibility of biology education within the classroom and teaching laboratory settings [CITE]. Many biology education research programs also evaluate the effectiveness of course-based undergraduate research experiences (CUREs) in increasing interest in science and providing a proper intervention to encourage higher representation of historically marginalized students within academia [CITE]. 

As a highly multidisciplinary field, bioinformatics presents a rare opportunity to understand how students learn and synthesize information spanning disparate fields and how teaching pedagogies unique to particular disciplines can be effective or ineffective. Students wishing to learn bioinformatics come from different backgrounds [CITE]. Additionally, the suggested core competencies for bioinformatics also differ depending on the professional level of the individual and desired skill set for the role they are in [CITE]. Teaching methods should then differ in how they approach students with a limited programming background, students with limited molecular biology knowledge, or students with experience in both fields separately but not integrated.  

%%%%%%%%%%%%%%%%%%%%%%%%%%%%%%%%%%%%%%%%%%%%%%%%%%%%%%%%%%%%%%%%%%%%%%%%%%%%%%%%
\section{Methods}
%%%%%%%%%%%%%%%%%%%%%%%%%%%%%%%%%%%%%%%%%%%%%%%%%%%%%%%%%%%%%%%%%%%%%%%%%%%%%%%%

\subsection{Graduate bioinformatics training at the University of California, San Diego}

The University of California, San Diego (UCSD) offers bioinformatics training at the undergraduate and graduate degree levels and the professional certification level at the university’s extension learning center [CITE]. Within this chapter, I will focus on the introductory bioinformatics training that I provided to masters, doctoral, and professional students across the courses and modalities provided in Table \ref{tab:course-table}. 

\begin{small}
    \tolerance=1 
    \emergencystretch=\maxdimen 
    \hyphenpenalty=10000 
    \hbadness=10000
    \begin{landscape} % this table is long, so it'll be multi-page landscape
        \begin{table}[]
            \caption{Bioinformatics courses taught at the University of California, San Diego}
            \label{tab:course-table}
            \begin{tabular}{p{.15\textwidth} p{.25\textwidth} p{.42\textwidth} p{.17\textwidth} p{.23\textwidth}}
            \hline
            \textbf{QUARTER} & \textbf{DEPARTMENT} & \textbf{COURSE NAME} & \textbf{MODALITY} & \textbf{WEBSITE} \\ \hline\hline \\

            Spring Quarter 2019 (Apr-Jun) & Scripps Institute of Oceanography (SIO) & SIOB 242C: Marine Biotechnology III, Introduction to Bioinformatics & In-Person & \textit{N/A} \\ \\ \hline \\ 
            
            Winter Quarter 2020 (Jan-Mar) & School of Medicine, Department of Cellular and Molecular Medicine & CMM 262/BIOM 262: Quantitative Methods in Genetics & In-Person & \href{https://github.com/biom262/cmm262-2020}{\texttt{cmm262-2020}} GitHub Repository \\ \\ \hline \\ 
            
            September 2020 (Before Fall Quarter 2020) & Jacobs School of Engineering & Bioinformatics \& Systems Biology Program Bootcamp & Online & \href{https://github.com/mragsac/BISB-Bootcamp-2020}{\texttt{BISB-Bootcamp-2020}} GitHub Repository \\ \\ \hline \\  

            Winter Quarter 2021 (Jan-Mar) & School of Medicine, Department of Cellular and Molecular Medicine & CMM 262/BIOM 262: Quantitative Methods in Genetics & Online & \href{https://github.com/biom262/cmm262-2021}{\texttt{cmm262-2021}} GitHub Repository \\ \\ \hline \\ 
            
            September Quarter 2021 (Before Fall Quarter 2021) & Jacobs School of Engineering & Bioinformatics \& Systems Biology Program Bootcamp & Online & \href{https://github.com/mragsac/BISB-Bootcamp-2021}{\texttt{BISB-Bootcamp-2021}} GitHub Repository \\ \\ \hline
            
            \end{tabular}
        \end{table}
    \end{landscape}
\end{small}

\subsubsection{SIOB 242C: Marine Biotechnology III, Introduction to Bioinformatics}

Conceptualized and taught by Theresa (Terry) Gaasterland, Ph.D. from the Scripps Institute of Oceanography (SIO), SIOB 242C is designed to give students an introduction to using high-performance computing systems to analyze real, primary RNA-sequencing data using command-line tools. In this class, there is a lecture once a week involving file manipulation and genomic data regular expressions in Unix, along with an accompanying take-home homework assignment. For this course, I acted as the only teaching assistant and hosted a weekly problem-solving session and office hours on an as-needed basis. Due to the small size of the graduate program at SIO, there were only ten students formally enrolled in the class. Additionally, the majority of students enrolled in the course had a background in marine biology without much computational experience.

\subsubsection{CMM 262/BIOM 262: Quantitative Methods in Genetics}
CMM 262 (also cross-listed as BIOM 262) is a required course for the UCSD Genetics Training Program and is designed to teach experimental and analytical approaches in modern genetics and genomics in several topic areas. I taught CMM 262 in Winter Quarter 2020 and Winter Quarter 2021 alongside Alon Goren, Ph.D. from the UCSD School of Medicine, and three other graduate students from the BISB Program. In this class, a guest instructor specializing in a particular subtopic of genetics presents two lectures to a class of approximately fifty biomedical sciences students. The teaching assistants for CMM 262 were responsible for coordinating guest faulty speakers, managing the distribution of course materials, grading course assignments and exams, and holding office hours for students. In the 2021 iteration of the class, I served as one of the lead teaching assistants. Additionally, due to the SARS-CoV-2 pandemic, this course was taught in-person and hybrid for 2020 and entirely online for 2021. Across both years, the majority of students enrolled in CMM 262 had a background in biomedical sciences without much exposure to computer programming.

\subsubsection{Bioinformatics \& Systems Biology Program Bootcamp}
Held every year during the week before the start of the academic year, the BISB Bootcamp is a student-run training course for incoming students to the BISB Doctoral Program. Through the BISB Bootcamp, incoming students are exposed to faculty research within the program and given a primer on topics in molecular biology, genetics, statistics, machine learning, computer science, and professional development meant to prepare them for their time in graduate school. As one of the course instructors, I was responsible for disseminating course materials to students before they arrived at UCSD, designing academic instructional modules, and logistical planning of the course. Due to the SARS-CoV-2 pandemic, the BISB Bootcamp was taught entirely online for 2020 and 2021. The students admitted to the BISB program are academically diverse, thus, students had varying degrees of exposure to computer programming and molecular biology.

\subsection{Publication of locally delivered bioinformatics course materials as open educational resources}

Open education is an educational movement founded on accessibility, transparency, and collaboration [CITE]. Open education aims to provide broader access to the learning and training provided through formal educational systems, such as the university environment [CITE]. To provide greater access to educational materials to individuals in various time zones worldwide, open education programs typically take advantage of online platforms to distribute content [CITE].  

Most course materials I developed for the bioinformatics courses I taught locally at UCSD were distributed as open educational resources (OERs) through the GitHub platform (Table \ref{tab:course-table}) to support the open education paradigm. OERs are educational resources (e.g., course materials, textbooks, multimedia applications) in the public domain that are openly available for instructors or students to retain, reuse, revise, remix, or redistribute without an accompanying need to pay royalties or licensing fees [CITE]. By distributing the materials through GitHub, I sought to increase the reach of the high-quality bioinformatics educational materials I created for UCSD while allowing people to revise, add, or remove course content as desired while using GitHub’s version-control feature for transparency of modifications. One of the fundamental guiding principles of open education is that everyone worldwide should have access to high-quality educational experiences and resources [CITE]. By publicizing the course content for CMM262 and the BISB Bootcamp, I aimed to eliminate barriers to this goal by reducing the high monetary costs of bioinformatics training and encouraging collaboration between scholars and educators in the field.

%%%%%%%%%%%%%%%%%%%%%%%%%%%%%%%%%%%%%%%%%%%%%%%%%%%%%%%%%%%%%%%%%%%%%%%%%%%%%%%%
\section{Results}
%%%%%%%%%%%%%%%%%%%%%%%%%%%%%%%%%%%%%%%%%%%%%%%%%%%%%%%%%%%%%%%%%%%%%%%%%%%%%%%%

Generally, bioinformatics courses often range in the course’s duration and the scope of the material covered (i.e., lecturing on single versus multiple topics). One of the most common formats includes short courses that cover a particular topic or analysis pipeline (e.g., evaluating single-cell RNA-sequencing analysis, genome-wide association studies, etc.) [CITE]. During graduate school, I taught a total of five comprehensive Python, R, and UNIX-based bioinformatics courses that covered multiple analysis pipelines related to transcriptomics, epigenetics, and population genetics (Table \ref{tab:course-table}). Within this section, I will discuss my strategies as a member of the teaching team for these courses to cater to the needs of students.

\subsection{Incorporating practical computational modules into course design}

There are many free bioinformatics online tutorials in the form of blog posts, GitHub-stored Jupyter Notebooks, and RMarkdown Books [CITE]. Unfortunately, biological and biomedical scientists sometimes find it difficult to directly apply these generic pipelines to their data, especially when they lack programming knowledge or the computational resources needed to run a particular analysis [CITE]. With any programming language, students require baseline skills in learning how to decode runtime errors and how to resolve these errors. The added complexity of data analytics requires that students analyzing biological data understand how the parameters for the tools they use impact their overall analysis and how these parameters balance with the system their study is conducted in. Thus, it can be difficult for students lacking the programming skills or theoretical biology background to apply off-the-shelf bioinformatics tools appropriately without guidance. 

Showcasing practical examples was important in ensuring students understood how to apply bioinformatics pipelines appropriately and in tempering expectations for bioinformatics as a whole. For example, when surveyed about a highly-interactive lecture on genome-wide association studies (GWAS) for CMM 262 taught in Winter 2021, students had high praise for the guest instructor: one student remarked in the free response section of the survey, “\textit{I really liked the coding exercises and doing them in real-time, it made me think through what was going on in the data…}” and another student mentioned, “\textit{The best part of the} [lecture] \textit{was the fact that the lines were not already filled so the class was a little bit more active…}” To foster students’ feelings of being active participants in lectures, we encouraged lecturers for CMM 262 to incorporate live programming in their lectures. Additionally, to ensure that students from SIOB 242C, CMM 242, and the BISB Bootcamp could apply knowledge from the courses to data produced from their present and future research labs, we specifically showcased well-known, existing community tools. Examples include \verb|samtools|, \verb|STAR|, \verb|seurat|, \verb|scanpy|, \verb|MACS2|, and others [CITE]. This ensured that after finishing our class, students would have access to a wealth of community resources and online forums with potential answers to their questions or answers to particular error prompts.

\subsection{Comparison of delivery methods for deploying bioinformatics assignments}

Many academic laboratories use high-performance computing (HPC) or cloud-based systems to analyze biological and biomedical datasets that cannot easily be processed on a laptop or desktop computer [CITE]. One example is the UCSD Triton Shared Compute Cluster (TSCC) housed at the San Diego Supercomputer Center (SDSC) [CITE]. TSCC is a condo cluster program that researchers can buy into through hardware purchases of computing nodes or by purchasing computing hours as account credits [CITE]. Despite the commonality of using Jupyter Notebooks for data exploration and visualization, academic labs can differ in how to access HPC or cloud-based computing systems based on ease of access, monetary constraints, or firewall requirements (i.e., medical data files protected by HIPAA have particular security requirements), monetary constraints, and ease of access [CITE]. Two common methods to access Jupyter Notebooks include command line-based and on-demand-based methods. However, both methods provide pros and cons for first-time bioinformatics learners. 

As part of SIOB 242C, CMM 242 taught in Winter Quarter 2020, and the BISB Bootcamp taught in September 2020, we worked with the San Diego Supercomputer Center (SDSC) to provide training accounts with enough credits for the entire quarter. Additionally, for the ChIP-sequencing analysis module taught in CMM 262 in Winter Quarter 2021, students were encouraged to complete bioinformatics pipelines similar to how you would on TSCC. With TSCC, students could learn additional skills in navigating the UNIX command line and using job scheduling systems to submit computational tasks. Students also learned how to customize software environments (e.g., conda environments) to cater to particular analysis pipelines. However, this additional layer between the student and course assignments introduced a larger learning curve toward the beginning of the course, especially for those that lacked prior programming experience. When students in CMM 262 taught in Winter 2021 were surveyed regarding their experiences learning to analyze ChIP-sequencing data through hands-on UNIX commands, many students felt the module was presented clearly. Upon being asked, “\textit{Did the lecturer present material clearly and understandably?}”, 33.3\% of students indicated Strongly Agree (9/27), 37\% indicated Agree (10/27), 18.5\% indicated Neither Agree nor Disagree (5/27), 11.1\% indicated Disagreed (3/27), and indicated 0.0\% Strongly Disagreed (0/27). But when reviewing the free response section of the survey, some students felt “\textit{…it was easy to fall behind…}” or “\textit{…the speed was too fast…}” whereas others felt that the pace “\textit{…could have been faster.}” The dichotomy in feedback in the free response section reflected the vast differences in technical background students had and their ability to follow along in the module. 

In contrast to 2020, for CMM 262 taught in Winter Quarter 2021 and the BISB Bootcamp taught in September 2021, we primarily used the JupyterHub platform to easily deploy data science notebooks to students that shared markdown text of lesson material alongside code blocks using bioinformatics tools. With JupyterHub, students could immediately jump into a particular course exercise without worrying about package installations or data transfers–-these were all aspects handled by the teaching staff and UCSD Educational Teaching Services. Unfortunately, upon coming out of the class, these students can face a larger barrier to pursuing bioinformatics in their research as they may have an idea of how to apply analysis platforms from their coursework but not how to set up or access the bioinformatics infrastructure they need. For example, when surveyed regarding the utility of Jupyter Notebooks in CMM 262 during Winter 2021, one student commented, “\textit{…since the code-along is being done on Jupyter Hub, I would like some additional resources or links on how to set up my machine (PC) for coding outside of the Hub.}” Additionally, several students remarked during office hours that a Jupyter Notebook-only approach without much practical, hands-on learning was less engaging. In a final course survey, there were responses such as, “\textit{…learning with just the} [Jupyter] \textit{notebook feels passive…}” and “\textit{…} [if] \textit{we had just executed} [ChIP-sequencing commands] \textit{in the notebook, I wouldn't have understood it as well, although I'm glad to have the notebook as a reference.}”

\subsection{Unifying students across diverse academic backgrounds in the classroom}

Designing courses for students from different backgrounds can be extremely challenging regardless of the subject taught [CITE]. Comprehensive introductory bioinformatics courses are no exception: the variation in course topics and the wide array of student academic backgrounds from typically non-intermixing fields make it difficult to design a course that can unify rather than alienate students in the classroom [CITE]. Students entering bioinformatics courses cover various specialties, from biological and biomedical sciences to the physical and computational sciences. One of the largest challenges in designing a comprehensive bioinformatics course is to develop in-class exercises and lectures that can unify the classroom rather than unintentionally isolate groups of students based on their knowledge gaps [CITE]. Typical knowledge gaps include programming, molecular biology, lab experience, and statistics. 

To cater to the diverse needs of students, my main goal was to enforce a culture of inclusivity of all academic and socioeconomic backgrounds to foster a less intimidating and safer classroom environment. Between SIOB 242C, CMM 262, and the BISB Bootcamp, there are distinct differences between the backgrounds of students. For example, students in SIOB 242C and CMM 262 have a primarily experimental biology background and often have limited programming exposure. On the other hand, students in the BISB Bootcamp have a highly diverse population of students that are often more computational without having extensive experience in genetics, genomics, and molecular biology techniques. For these two groups, the teaching style varies to accommodate their unique backgrounds. 

\subsection{Teaching students with biological backgrounds to adopt a growth mindset in learning bioinformatics}

In teaching students that have limited computational experience, my initial goal is to make computer science and computing more accessible and less intimidating, especially for groups of students historically excluded from these subjects. Due to inequitable access to computer science education before college, many students feel unprepared for or unsuitable for introductory computer science coursework [CITE]. Psychological roadblocks–-such as stereotype threat and imposter syndrome–-can also contribute to students’ perceived potential success in computer science [CITE]. When I teach introductory bioinformatics courses, I address these concerns to boost students’ confidence and to foster a classroom environment where these concerns can be discussed openly with other students and the teaching staff. Additionally, I explicitly state that prior programming experience is not required to succeed within introductory bioinformatics courses to eliminate preconceived notions about required background knowledge before instruction takes place.

Stereotype threat is when an individual feels at risk of confirming negative stereotypes about the group of which they are a member [CITE]. Situational factors that contribute to stereotype threat include the task’s difficulty at hand, the belief that the task measures their abilities, and the relevance of the stereotype to the task [CITE]. Stereotype threat is believed to be a psychological barrier to students’ engagement in computer science due to its ability to contribute to diminished confidence, poor performance, and loss of interest in the field, especially for minoritized students [CITE]. While computer science courses tend to attract more men and more white and South Asian or East Asian students, biological science courses comparatively attract more women and more Latine and Black students [CITE]. Thus, as an instructor, I try to be welcoming and compassionate towards the women and non-binary students and students from minoritized groups that enter the classroom. In teaching bioinformatics, I address students’ concerns as they arise to ensure retention and foster interest in the field.

One of the main points of concern I observed from students in SIOB 242C and CMM 262 was their inability to learn how to program late in their academic careers. Thus, the primary strategy I employed to help these students was to encourage the adoption of a growth mindset. One theory of intelligence holds that people can be categorized into two groups based on their implicit beliefs about their ability to learn [CITE]. People with a fixed mindset believe that learning ability is innate, whereas people with a growth mindset believe knowledge can be acquired through effort and studying [CITE]. Computer science is a difficult subject for first-time learners due to (i) the steep initial learning curve in learning a new language, (ii) the detail-oriented nature required to meet syntaxial requirements, and (iii) the constructive nature of computer science as a discipline [CITE]. It is important to address each of these difficulties during instruction to encourage the development of a growth mindset in the classroom. 

Most of my teaching success was derived from live programming to solve bioinformatics problems during course instruction. Because of the steep initial learning curve involved in computer science, concepts should be introduced slowly and explained explicitly [CITE]. Within SIOB 242C and CMM 262, I incorporated live programming in my teaching to naturally explain new computer science concepts (e.g., variable declaration, for and while loops, conditional expressions) as they pertained to solving a bioinformatics problem in real-time. In live programming, I aimed to demystify the black box that bioinformatics can often feel like and provide a practical example of how computational concepts can be easily applied to students’ work outside the classroom to encourage engagement with the material [CITE]. For example, when teaching a learning module on basic statistics for CMM 262 during Winter 2021, a common point of feedback was that students “\textit{…found the practical examples in notebooks extremely helpful.}” Several students also felt motivated to program on their own, and one comment indicated that “\textit{…as someone who is brand new to programming it might be nice if there were a few brief practice problems we could try out on our own and see posted answer keys later…}” Computer programming requires that people be meticulous about noticing syntaxial errors in particular programming languages [CITE]. Through live programming, I was also able to touch on the importance of being detail-oriented when it comes to computer programming [CITE]. During live programming demos, students often pointed out errors in my programming and suggested modifications to my code to make things run successfully. Finally, we could trace through errors in programming logic together as a class, enforcing a unified community spirit in the classroom. 

\subsection{Reducing information overload in teaching bioinformatics to computational students}

Introductory biology courses typically cover a multitude of topics, and it is well known that students at the secondary and undergraduate levels face difficulties in learning biological concepts [CITE]. For instance, biology classes have overloaded curricula and cover abstract topics [CITE]. These two factors combined often lead to preconceived notions of rote memorization being the defining feature of biology as a whole [CITE]. Within the BISB Bootcamp, there was a larger proportion of computational students compared to SIOB 242C and CMM 262. These students had engineering, physical, or computer science backgrounds but no extensive experience with molecular biology or genetics. My primary goal in teaching these students was to teach core biology concepts that present themselves in commonly-discussed bioinformatics problems to reduce cognitive overload. 

Cognitive overload or information overload occurs when you are exposed to more details than you can process at any given time [CITE]. Additionally, cognitive overload can manifest as mental fatigue, reduced attention span, and behavioral changes [CITE]. Similarly to teaching experimental biology students, I employed one strategy to reduce cognitive overload: slowly introducing biology terminology and concepts as they become relevant to the bioinformatics problem. This teaching method can also be seen through a widely-taken bioinformatics Coursera course series developed by Pavel Pevzner and Philip Compeau, as they introduce questions in biology that can use computational methods for answer generation [CITE]. 

Within the BISB Bootcamp, this teaching method was used when students were presented with the problem of looking for transcription factor binding sites within a sequence. In this example, students were introduced to several ideas in the following order: 

\par\noindent\dotfill

\begin{enumerate}
    \item \textbf{Providing Background Information on the Biological Problem}
    \begin{enumerate}
        \item Consider a sentence as a string of words made up of individual letters or characters. 
        \item Also, consider that genetic information is contained within all the cells of the body as DNA.
        \begin{enumerate}
            \item DNA consists of the nucleic acids adenine (\verb|A|), thymine (\verb|T|), cytosine (\verb|C|), and guanine (\verb|G|).
        \end{enumerate}
        \item DNA can be represented as a string of different characters representing the nucleic acids.  
        \item There are regions of DNA that transcription factor proteins can bind to through the recognition of short DNA sequences (e.g., \verb|GATA|) to activate particular genes.
        \begin{enumerate}
            \item These regions are otherwise known as transcription factor binding sites.
        \end{enumerate}
    \end{enumerate}
    \item \textbf{Defining the Biological Problem and the Associated Computational Problem}
    \begin{enumerate}
        \item If we consider DNA as a sentence, these binding regions can be the words in our sentence that we’re trying to understand the meaning of (e.g., how come certain transcription factors activate certain genes, and are there any patterns?). 
        \item To further understand gene activation, we need to be able to recognize where transcription factor binding sites are within a DNA sequence!
        \item Looking for transcription factor binding sites within a DNA sequence can be considered a computational “search” problem of looking for a substring within a string! 
    \end{enumerate}
    \item \textbf{Developing a Bioinformatics Solution}
    \begin{enumerate}
        \item We can define variables that represent the DNA sequence and the binding site. 
        \item Next, loop through each position in the DNA sequence to see if the transcription factor binding site matches the start of the sequence. 
        \begin{enumerate}
            \item If the site is found, we’ve successfully identified the location of a binding site! 
            \begin{enumerate}
                \item We can save the position with a match and then continue to the next position in the sequence to look for more binding site matches. 
            \end{enumerate}
            \item If we have looked at all positions in the entire DNA sequence and haven’t found a binding site match, then the site does not exist. 
        \end{enumerate}
    \end{enumerate}
\end{enumerate}

\par\noindent\dotfill

When prompted with an optional survey to score this module from a range of one for “\textit{Uninformative}” to five for “\textit{Transcendent},” 37.5\% of students (6/16) rated the module with a score of five, 6.3\% of students (1/16) rated the module with a score of four, 37.5\% of students (6/16) rated the module with a score of three, and 18.8\% of students (3/16) left the field blank. In the free response section to provide feedback for this module, students were relatively satisfied with the teaching format, commenting, “[the interactive module] \textit{was definitely needed for the rest of the week,}” “\textit{I think it was useful and I got more out of the session I attended than I would have from the other} [lecture without programming in biology],” and “\textit{It was very useful.}” Additionally, one student with a larger background in computer science commented, “\textit{…I was able to do the CS project/presentation} [with] \textit{no problem} [while interacting] \textit{with the biology live.}” 

While this particular example relies on prior molecular biology knowledge of DNA and proteins, we reduced the amount of background information required to understand the overall goal of this example and the applicable bioinformatics problem. In particular, we abstracted the concept of gene activation for the audience by not mentioning other parts of the system, such as transcriptional cofactors, enhancers, promoters, or genome methylation. Students were then able to easily recognize the value of computational methods when integrated with molecular biology, and our teaching methodology helped spark interest in students’ interest in theoretical molecular biology across various topics. For example, one student provided the following comment, “\textit{As someone with no biology background, it did go a bit over my head. However, I still found it useful to hear about different techniques even if I didn't fully understand them … I had a great discussion with} [the Bootcamp instructors] \textit{at the end of the lecture about possibly working in a wet lab.}” 

\subsection{Using interactive teaching pedagogies to encourage student participation} 

The oldest teaching pedagogy is known to be “teacher-centric,” where the instructor lectures students who enter the classroom as a \textit{tabula rasa}, expected to passively receive the knowledge being disseminated [CITE]. Under this paradigm, the instructor is the core regulator of knowledge in the classroom: they do most of the talking, set the rules and learning goals, and drive the direction of follow-up discussions [CITE]. Recently, classrooms–-especially juvenile classrooms--have started adopting a “student-centric” pedagogy [CITE]. In this environment, students control the direction of learning through collaborative discussions with their peers after being given the required conditions and tools by the instructor [CITE]. While teacher-centric and student-centric methods fall at opposite ends of the spectrum, instructors use varying proportions of each methodology, known as “interactive teaching” [CITE].

Bioinformatics is based on technological advancements in biology and, thus, relies heavily on access to a computer, especially for data analytics [CITE]. For bioinformatics courses focused on data analysis rather than algorithmic design, we can easily incorporate interactive teaching into course lectures [CITE]. Course lectures were modified in real-time based on student feedback in SIOB 242C, CMM 262, and the BISB Bootcamp. Each concept was taught as a “block” consisting of four components: (i) a molecular biology concept (e.g., genome sequences), (ii) an open question concerning the concept presented (e.g., comparing genomes), (iii) a parallel computer science concept (e.g., string comparisons), and (iv) an example computational solution to the question (e.g., genome/string alignment with dynamic programming). By being upfront with the interdisciplinary nature of bioinformatics problems, students of all backgrounds were engaged in asking questions during course instruction and providing solutions to questions provided during live programming demonstrations. 

\subsection{The impact of COVID-19 on teaching university-level bioinformatics courses in 2020 and 2021}

In recent years, universities have adopted online educational tools into regular instruction to provide greater accessibility to course materials, external resources, and grading information online. For example, many universities use the Canvas web-based learning management system (LMS) [CITE]. Many engineering and computational courses use standardized platforms for student communication and course assessments, such as Piazza and GradeScope [CITE]. In addition to LMS platforms, some universities have started exploring “flipped classroom” formats in which students encounter lecture material independently before dedicating all in-person instructional time to discussion-like sessions [CITE]. However, towards the end of 2020, this gradual process of virtualizing traditional in-person courses was greatly accelerated by the high aerosol transmissibility of the SARS-CoV-2 virus [CITE]. 

The emergency of the Coronavirus Disease 2019 (COVID-19) crisis forced instructors worldwide to translate their in-person courses into virtual environments, introducing difficulties in promoting interaction between students and instructors, especially in a medium unfamiliar to many. Despite bioinformatics’ reliance on virtual resources and computers as a field, many still faced challenges translating successful in-person courses to online-only mediums. In addition to the commonly observed issue of student engagement, one of the largest challenges in CMM 262 and the BISB Bootcamp was losing important in-person interactions in teaching and learning programming for the first time [CITE]. For example, assisting students in live programming or in-class pair-programming sessions was more difficult when they ran into individual errors with the coding module. It was also challenging to facilitate small group discussions. While it is easy to walk up to students to help them with technical difficulties during in-person instruction, we were forced to take advantage of Zoom's "breakout room" feature to assist these students. One student from CMM 262 taught in Winter 2021 commented on the interactive ChIP-sequencing analysis pipeline module: "\textit{This module would be one that would benefit from in-person instruction, because it was easy to fall behind during the coding segment. I didn't want to interrupt the class to slow down and I would have been more comfortable asking a TA or a neighbor in a physical classroom.}" Ultimately, it was difficult to assist students with conceptual or technical difficulties during lecture time. Often, these students would approach the teaching team during office hours to resolve any issues. 

Fortunately, there were some benefits to moving the course entirely online. Students, for example, could review recorded content during their own time. Two students from the Winter 2021 iteration of CMM 262 commented, "\textit{I think the recorded Zoom lectures help though, since I definitely needed to rewatch some parts}" and "\textit{…since the lectures are recorded, I am able to go back and go through it at my own pace, which is really helpful and appreciated!}" Another student commented on the same course, "\textit{For an online format, the course worked well when it came to being able to access the lecture recordings with captions since it can be hard to sit through an online lecture without them. I felt that the course was not adapted for longer lectures since I was experiencing Zoom fatigue and could not hold my attention for more than an hour (maybe note-taking-friendly formats or shorter, more frequent lectures may help).}" Properly deploying synchronous, practical bioinformatics classes requires instructors to consider how online mediums such as Zoom will impact students' learning experience. While we encountered logistical difficulties in interacting with students through Zoom breakout rooms or combating Zoom fatigue, the transition to online education, accelerated by the SARS-CoV-2 virus, underscored the possibility of making bioinformatics education more accessible to a broader audience. 

%%%%%%%%%%%%%%%%%%%%%%%%%%%%%%%%%%%%%%%%%%%%%%%%%%%%%%%%%%%%%%%%%%%%%%%%%%%%%%%%
\section{Conclusion}
%%%%%%%%%%%%%%%%%%%%%%%%%%%%%%%%%%%%%%%%%%%%%%%%%%%%%%%%%%%%%%%%%%%%%%%%%%%%%%%%

We are currently in a transition period in how we approach undergraduate Biology education from one that takes a surface-level approach in introducing bioinformatics analyses in one-off modules to one that integrates traditional computational courses into the canonical curriculum. While these changes will ultimately benefit the next generation of scientists in analyzing the large-scale biological datasets of the future, there is a need to address the knowledge gap for graduate students and other professional scientists of the present. While it is important to consider incorporating practical course modules into bioinformatics and balance the amount of material to include within bioinformatics classes, one of the largest considerations is the background of the students being taught.

Students wishing to learn bioinformatics later in their careers often come from various specialties spanning biological and biomedical sciences to the physical and computational sciences. Thus, one of the largest challenges in designing a comprehensive bioinformatics course is balancing these diverse backgrounds with designing course material that does not isolate students based on their knowledge gaps in theoretical biology, programming, and statistics. To teach bioinformatics to an academically diverse classroom, incorporating course materials that incorporate aspects of everybody’s background help create a common ground for people to grow. Within SIOB 242C, CMM 262, and the BISB Bootcamp, showcasing computer science concepts of data types and looping in the context of analyzing genomic sequences proved successful in teaching biological and biomedical sciences students while cementing core instructional concepts and reducing the psychological barrier of stereotype threat. Slowly introducing theoretical biology in the context of interesting computational problems also successfully taught computational students without inflicting information overload. Balancing course content with students’ learning abilities makes it possible to unify the classroom without leaving people behind. Additionally, introducing practical bioinformatics examples through student-paced live programming helps make bioinformatics accessible to new audiences and encourages an inclusive environment for all academic backgrounds.

%%%%%%%%%%%%%%%%%%%%%%%%%%%%%%%%%%%%%%%%%%%%%%%%%%%%%%%%%%%%%%%%%%%%%%%%%%%%%%%%
\section{Acknowledgments}
%%%%%%%%%%%%%%%%%%%%%%%%%%%%%%%%%%%%%%%%%%%%%%%%%%%%%%%%%%%%%%%%%%%%%%%%%%%%%%%%

I would like to thank Niema Moshiri, Clarence Mah, and Emma Farley for their thoughts and helpful feedback in writing this chapter. Additionally, in teaching introductory bioinformatics courses at UCSD, I learned from the students and from the other instructors I worked with in developing the course materials, lectures, and assignments. SIOB 242C, CMM 262, and the BISB Bootcamp courses would not have been successful without their dedicated support. 

Firstly, I am grateful to Alon Goren, Daniela "Dana" Nachmanson, Clarence Mah, Eric Kofman, Pratibha Jagannatha, and all of our guest instructors for being wonderful and flexible members of the teaching team for CMM 262 taught during the Winter Quarters of 2020 and 2021. Through teaching this class, I fine-tuned my knowledge of diverse bioinformatics pipelines and met many of the wonderful students in the Biomedical Sciences (BMS) graduate program. 

Next, I would like to thank Owen Chapman, Cameron Martino, Mike Cuoco, and Lauryn Bruce for their help in co-teaching the BISB-Biomedical Sciences (BMS) Joint Program Bootcamp in September 2020 and the BISB Bootcamp in September 2021. Planning a student-run Bootcamp in the limbo of the early COVID-19 pandemic on top of doing my thesis research was especially stressful, and I’m thankful to have had Owen and Cameron by my side to navigate the uncertainties of whether or not we would be able to teach a 50-person, primarily experimental class on how to code on the command line in a week. I would also like to thank the numerous guest student instructors from both Bootcamp sessions who took the time outside of research obligations to teach their peers various skills needed to survive both the personal and professional aspects of graduate school. This included Alexander Wenzel, Gibraan Rahman, Clarence Mah, Adam Officer, and George Armstrong from the BISB program, as well as Alex Tankka, Sara Elmsaouri, Danielle Schafer, Maya Gosztyla, Noorsher Ahmed, and Margaret Burns from the BMS program. Without these dedicated individuals, we would have never been able to cover as much material as we did, and thus you have my thanks! 

Finally, I would like to express my deepest gratitude to Terry Gaasterland for being an encouraging bioinformatics education mentor since I first took a variant of what is now SIOB 242C as an undergraduate student during Spring Quarter 2015. That class, SIO 190, sparked my interest in bioinformatics as a discipline and set the foundation for how I approach analyses today. I would also like to thank her for trusting me to assist in teaching SIOB 242C for the Spring Quarter 2017 course as a senior undergraduate student and then again during Spring Quarter 2019 as a first-year Ph.D. student. Without those experiences, I would not have been able to develop my passion for developing educational materials for students or experiment with incorporating new teaching methods into the classroom. It was also Terry’s idea that I include this chapter within my thesis; for that, I will always be forever thankful that these reflections about teaching were not lost to time.
