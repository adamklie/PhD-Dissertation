\chapter{Understanding \textit{Ciona intestinalis} gastrulation at single-cell resolution}

%%%%%%%%%%%%%%%%%%%%%%%%%%%%%%%%%%%%%%%%%%%%%%%%%%%%%%%%%%%%%%%%%%%%%%%%%%%%%%%%
\section{Introduction}
%%%%%%%%%%%%%%%%%%%%%%%%%%%%%%%%%%%%%%%%%%%%%%%%%%%%%%%%%%%%%%%%%%%%%%%%%%%%%%%%

Embryonic development begins upon the fertilization of an egg by a sperm cell to become a single-cell zygote, which continues through many stages of cell division to form a functional organism [CITE] eventually. Developmental processes are finely orchestrated by gene regulatory networks (GRNs), collections of genes that interact with each other to mediate gene expression. GRNs govern the necessary embryonic axis formation and body plan patterning processes required for proper development [CITE]. Although genetics and experimental embryology have dissected the major transcription factors and secreted signaling molecules involved in the specification of early cell lineages, the processes governing development involve many circuits beyond the well-known factors [CITE]. Thus, there is a continued need to explore the mechanisms involved in development to understand how deficiencies in cell fate specification contribute to developmental disease. Historically, gene expression studies have been limited to analyzing pooled populations of cells to obtain sufficient RNA for analysis despite the importance of cell heterogeneity in organ development [CITE]. Fortunately, advances in genomic technologies have allowed developmental biologists to assess the early gene expression events associated with fate specification in single cells [CITE]. Through single-cell RNA sequencing (scRNA-seq), we can now evaluate the RNA expression of every gene at single-cell resolution. In this chapter, I used scRNA-seq to explore early organ formation in the urochordate, \textit{Ciona intestinalis type A} (also known as \textit{Ciona robusta} or \textit{Ciona}), to understand the GRN governing notochord development during gastrulation.

Gastrulation is an early, formative developmental process that involves the reorganization of an embryo from a one-dimensional layer of epithelial cells (blastula or blastocyst) into a multi-layered, multi-dimensional structure (gastrula) [CITE]. It results in the formation of the major germ layers in the developing embryo (e.g., endoderm, ectoderm, and mesoderm) that act as precursors to all embryonic tissues, as well as the establishment of the dorsal/ventral and anterior/posterior axial orientations of the embryo [CITE]. After forming the major germ layers, the embryo is primed for key organ and structure formation. The phylum Chordata is a large division of the animal kingdom that includes vertebrates, tunicates, and cephalochordates [CITE]. All chordate embryos share, among a few other hallmarks, a defining structural feature known as the notochord that forms during gastrulation that is present during some or all of their life cycle [CITE]. The notochord is a hollow tube of mesodermal origin extending from the anterior to the prechordal plate [CITE]. It is a flexible, midline cartilaginous rod of tissue found in very close connection with the ventral-most region of the neural tube. Beyond its structural role, the notochord plays an indispensable role in the formation of the neural tube through the secretion of various developmental morphogens, including \textit{sonic hedgehog} (\textit{shh}) [CITE]. The intricate relationship between the notochord and the formation of other key structures, such as the neural tube, renders it necessary to understand notogenesis to treat notochord-derived disorders and defects. 

Within vertebrates, the notochord is a transient anatomical structure only present in the early embryo. Notochord-derived abnormalities can be traced to stress on the pathways responsible for notochord cell maintenance in adulthood or to remnants of the notochord that fail to regress during early development. The remnants of the notochord constitute the nucleus pulposus, the innermost compartment of the intervertebral discs [CITE]. Within the nucleus pulposus, notochord cells secrete extracellular matrix (ECM) molecules to form a proteoglycan-rich and gelatinous matrix that acts as the cushioning infrastructure responsible for the shock-absorption properties of the intervertebral discs [CITE]. These properties are necessary for general movement and flexibility of the backbone in vertebrates [CITE]. Degeneration of notochordal cells in the nuclei pulposi causes the onset of intervertebral disc degeneration and consequent back pain, the leading cause of disability in the adult population worldwide [CITE]. Thus, many groups have focused on dissecting the factors important for notogenesis to identify potential therapeutic agents to limit or reduce the symptom-causing pathologies of intervertebral disc degeneration by targeting pathways inducing structural disruption or inflammation [CITE]. Another notochordal defect includes chordomas, a rare type of bone sarcoma that represents about 1\% to 4\% of primary bone tumors [CITE]. While the mechanistic knowledge of chordoma formation is limited, there is evidence that they are derived from embryonic remnants of the notochord [CITE]. For example, long before it was proposed as a diagnostic marker for chordomas, brachyury was identified as a regulator for notogenesis and as a general biomarker for the notochord and notochord-derived tumors [CITE]. Brachyury is a highly conserved T-box transcription factor that helps promote cell movement and adhesion, which are fundamental for morphogenesis and tumorigenesis [CITE]. With the fundamental role of brachyury in notochord development, further research into the factors involved in notogenesis is important to better understand whether aberrant activation of notochord GRNs contributes to chordomagenesis.

The marine tunicate \textit{Ciona} is a member of the subphylum Urochordata and is thought to represent the simplest and most primitive chordate body plans [CITE]. While \textit{Ciona} has been extensively studied, there is still much we can delineate from comparing \textit{Ciona} cell fate determination pathways to other chordate species, especially concerning notochord specification [CITE]. In a previous study, Cao et al. developed a single-cell transcriptional atlas spanning the onset of gastrulation through the swimming tadpole stage in \textit{Ciona}. Within this study, they were able to construct virtual cell-lineage maps and gene networks for 41 neural subtypes that comprise the larval nervous system [CITE]. Other single-cell studies performed in tunicates have also proved successful in studying lineage specification in other cell types [CITE]. As various groups have demonstrated the feasibility of performing atlas-scale single-cell methods in \textit{Ciona} and other tunicates, we used scRNA-seq to generate a comprehensive single-cell gene expression atlas spanning the onset of gastrulation to study the GRNs dictating notochord fate specification. 
