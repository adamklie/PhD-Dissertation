\begin{dissertationintroduction}

    %%%%%%%%%%%%%%%%%%%%%%%%%%%%%%%%%%%%%%%%%%%%%%%%%%%%%%%%%%%%%%%%%%%%%%%%%%%%
    \section{The genome and the central dogma of biology}
    %%%%%%%%%%%%%%%%%%%%%%%%%%%%%%%%%%%%%%%%%%%%%%%%%%%%%%%%%%%%%%%%%%%%%%%%%%%%

    In every human cell, nearly two meters of DNA are compacted into the nucleus, an organelle only a few micrometers wide. This molecule encodes the instructions to build and regulate an entire organism, using just four chemical bases: adenine (A), cytosine (C), guanine (G), and thymine (T). The efficiency of this system is one of the many marvels of molecular biology.

    The classical framework for understanding how this information is used is the central dogma of molecular biology, proposed by Francis Crick in 1958 {}. It posits a directional flow of information: DNA is transcribed into RNA, which is then translated into protein. Proteins in turn carry out the structural, enzymatic, and regulatory functions necessary for life. While numerous exceptions and refinements to the central dogma have since been discovered, the core concept continues to underpin molecular biology today.

    However, this model alone is insufficient to explain a fundamental observation in genomics: the number of protein-coding genes does not scale with organismal complexity {}. Humans, for example, have roughly 20,000 protein-coding genes, fewer than some plants and not dramatically more than worms or flies. This suggests that biological complexity arises not solely from the repertoire of proteins encoded in the genome, but from how, when, and where those proteins are expressed {}.

    One hypothesis that gene regulation is a primary driver of biological complexity. Gene regulation refers to…

    There are many layers … gene regulation but focus of this thesis is at the level of transcription. We point the reader to excellent reviews {}...

    Transcription is th process by which

    Gene expression is controlled in a highly dynamic, context-specific manner through …

    This thesis explores the mechanisms and principles of gene regulation through the lens of predictive modeling. By using computational models trained on genomic and epigenomic data, we can begin to uncover the rules that govern gene expression across contexts, cell types, and perturbations, to make predictions… and to design. The remainder of this introduction provides an overview of the molecular mechanisms of transcriptional regulation, followed by a discussion of the computational tools that have emerged to model them.

    %%%%%%%%%%%%%%%%%%%%%%%%%%%%%%%%%%%%%%%%%%%%%%%%%%%%%%%%%%%%%%%%%%%%%%%%%%%%
    \section{Next-generation sequencing}
    %%%%%%%%%%%%%%%%%%%%%%%%%%%%%%%%%%%%%%%%%%%%%%%%%%%%%%%%%%%%%%%%%%%%%%%%%%%%

    Assays presented in this thesis are impossible to understand without an appreciation for next-generation sequencing

    This section should provide a lite overview and refer the reader to more in depth resources
    
    %%%%%%%%%%%%%%%%%%%%%%%%%%%%%%%%%%%%%%%%%%%%%%%%%%%%%%%%%%%%%%%%%%%%%%%%%%%%
    \section{Mechanisms of transcriptional regulation}
    %%%%%%%%%%%%%%%%%%%%%%%%%%%%%%%%%%%%%%%%%%%%%%%%%%%%%%%%%%%%%%%%%%%%%%%%%%%%

    Gene expression is controlled by a complex interplay of molecular mechanisms that operate across multiple layers of the genome and epigenome. The regulation of transcription determines whether and how a gene is expressed in a given cell. This regulation is shaped by both the local DNA sequence and its broader chromatin environment.

    \subsection{Chromatin Accessibility and Epigenomic Modifications}

    In eukaryotic cells, DNA is tightly packaged. At the lowest level, ~147 bp of DNA is wrapped around an octamer of histone proteins to form a nucleosome {ref}. Other levels of packaging

    This compaction both protects DNA and restricts its accessibility to interact with... Large regions of closed chromatin, termed heterochromatin, are densely nucleosome-bound and transcriptionally inert, whereas expressed genes and regulatory elements such as promoters {}, enhancers {}, and insulators {} are typically depleted of nucleosomes on euchromatin. The result is a highly dynamic landscape of “chromatin accessibility” that reflects the regulatory state of the genome. [BETTER TRANSITION] Nucleosome occupancy, spacing, and the presence of linker histones vary across the genome, producing a continuum of chromatin states ranging from tightly compacted (inaccessible) to permissive {}. 

    Chromatin remodelers and epigenome-modifying enzymes play key roles in establishing and maintaining accessible or repressive chromatin configurations. For example, histone acetyltransferases (HATs) and histone deacetylases (HDACs) can respectively add or remove acetyl groups from histone tails, altering nucleosome stability and influencing transcriptional accessibility {}. Similarly, histone methyltransferases (HMTs) and demethylases add or remove methyl groups at specific lysine residues, with distinct consequences depending on the site and context—for instance, H3K4me3 is associated with active promoters {}, whereas H3K27me3 marks repressed chromatin {}. In parallel, DNA methyltransferases (DNMTs) catalyze the addition of methyl groups to cytosines in CpG dinucleotides, often reinforcing long-term gene silencing {}. These modifications can act as recruitment signals for chromatin-binding proteins, such as methyl-CpG binding domain (MBD) proteins, which further compact chromatin or recruit corepressors.

    A number of high-throughput sequencing–based assays have been developed that profile chromatin accessibility genome-wide. These assays typically rely on the principle that regions of open chromatin are more susceptible to enzymatic activity—either cleavage or chemical modification—than closed, nucleosome-bound regions {}. Among the earliest developed assays was DNase I hypersensitive site sequencing (DNase-seq), which identifies accessible regions by detecting preferential cleavage by DNase I {}. FAIRE-seq (Formaldehyde-Assisted Isolation of Regulatory Elements) uses crosslinking and organic extraction to enrich for nucleosome-depleted DNA {}. More recently, ATAC-seq (Assay for Transposase-Accessible Chromatin using sequencing) has become the predominant method due to its low input requirements, speed, and scalability {ref}. In ATAC-seq, a hyperactive Tn5 transposase simultaneously cleaves accessible DNA and inserts sequencing adapters. The resulting fragments reflect nucleosome-free regions and can be sequenced via next-generation sequencing.

    To assess DNA methylation, one of the most widely used techniques is Whole Genome Bisulfite Sequencing (WGBS) {}. In this method, bisulfite treatment converts unmethylated cytosines to uracils, which are read as thymines during sequencing, while methylated cytosines remain unconverted. By comparing treated and untreated sequences, WGBS yields single-base resolution maps of methylation across the genome. Although more costly and coverage-intensive than accessibility assays, WGBS provides a comprehensive view of the DNA methylation landscape, which plays a key role in maintaining transcriptional repression and establishing cell identity {}.

    \subsection{Chromatin Accessibility and Epigenomic Modifications}

    \subsection{Transcription factor binding}

    \subsection{Enhancers and cis-regulatory elements}

    \subsection{Chromatin Accessibility and Epigenomic Modifications}

    \subsection{Chromatin conformation and long-range interactions}

    \subsection{Advances in single-cell assays for regulatory genomics}

    %%%%%%%%%%%%%%%%%%%%%%%%%%%%%%%%%%%%%%%%%%%%%%%%%%%%%%%%%%%%%%%%%%%%%%%%%%%%
    \section{Predictive models of transcriptional regulation}
    %%%%%%%%%%%%%%%%%%%%%%%%%%%%%%%%%%%%%%%%%%%%%%%%%%%%%%%%%%%%%%%%%%%%%%%%%%%%

    Disentangling the complexity of transcriptional regulation requires building computational models—formal representations that link molecular features, such as DNA sequence or chromatin state, to functional readouts like transcription factor binding, chromatin accessibility, or gene expression. Among the many ways to approach model building, one particularly powerful strategy is by predicting.

    When models accurately predict observed functional readouts, they do more than simply reproduce the data. When properly interpreted, they can also reveal which features are informative, how they interact, and where existing annotations may fall short. In the remainder of this section, I introduce the predictive modeling framework known as supervised learning, and then focus on a class of models that has gained significant traction in regulatory genomics: those that take DNA sequence as input and predict regulatory activity. These “sequence-to-function” models form the methodological backbone of this thesis.

    \subsection{Learning by predicting: supervised models in regulatory genomics}

    \subsection{Sequence-to-function models}

    \subsection{Applications: prediction, design, and interpretation}

    %%%%%%%%%%%%%%%%%%%%%%%%%%%%%%%%%%%%%%%%%%%%%%%%%%%%%%%%%%%%%%%%%%%%%%%%%%%%
    \section{Aims and Scope of this thesis}
    %%%%%%%%%%%%%%%%%%%%%%%%%%%%%%%%%%%%%%%%%%%%%%%%%%%%%%%%%%%%%%%%%%%%%%%%%%%%

    This thesis explores how predictive models can be used to better understand, interpret, and manipulate transcriptional regulation from DNA sequence. Its contributions are both methodological and biological: I develop unified tools for training and interpreting machine learning models in regulatory genomics, and apply these tools across multiple biological contexts to illustrate their utility.

    Universal software for machine learning in regulatory genomics

    A general-purpose software toolkit for training and interpreting machine learning models in regulatory genomics. This tool abstracts and standardizes common workflows, enabling rapid development, interpretation, and reuse of sequence-to-function models across diverse datasets and prediction tasks.

    Application 1: TFBS syntax-driven design of neural-specific enhancers in vivo

    Chapter 3 explores how TFBS syntax governs enhancer activity by developing a predictive model trained on synthetic and genomic enhancer sequences. I use this model to identify organizational rules of neural-specific enhancers and validate predictions using in vivo assays, illustrating how sequence-based models can support mechanistic insight and enhancer engineering.

    Application 2: Deciphering the impact of genomic variation on gene regulation in pancreatic β-cell organoids

    Chapter 4 focuses on variant interpretation in a disease-relevant system. I use sequence-based models to quantify the impact of noncoding genetic variants on chromatin accessibility in pancreatic β-cell organoids, integrating single-cell multiomic data and model interpretation methods to link sequence variation to regulatory disruption. Interpret the effects of genomic variation on chromatin accessibility

    Together, these chapters highlight the versatility of predictive modeling approaches in regulatory genomics and demonstrate how sequence-to-function models can be applied to understand biology, interpret genetic variation, and guide experimental design.

\end{dissertationintroduction}
