\begin{dissertationepilogue}
    %%%%%%%%%%%%%%%%%%%%%%%%%%%%%%%%%%%%%%%%%%%%%%%%%%%%%%%%%%%%%%%%%%%%%%%%%%%%
    \section{Conclusion}
    %%%%%%%%%%%%%%%%%%%%%%%%%%%%%%%%%%%%%%%%%%%%%%%%%%%%%%%%%%%%%%%%%%%%%%%%%%%%
    I started this work with a quote from Lewis Wolpert, "It is not birth, marriage, or death, but gastrulation that is the most important time of our lives." Indeed, gastrulation is a critical step in embryonic development. Gastrulation is when the primordial germ layers are specified, the embryonic axes manifest, and the embryo alters its morphology for the first time. Within chordates, the formation of the primary germinal layers-the ectoderm, mesoderm, and endoderm-requires meticulous control of individual cell and collective tissue behaviors with regard to space and time \cite{ghimire2021,balmer2016,winkley2020,solnica-krezel2012}. Thus, it is crucial to study the contents of a cell and the active regulatory factors at this stage to understand how defects in this machinery lead to congenital disabilities and disease. 
    
    One structure that emerges during gastrulation is the notochord, a rod-like, cartilaginous skeleton of mesodermal origin that defines chordates. The notochord serves as a signaling center for the embryonic midline and becomes an integral part of the vertebrate backbone as the nucleus pulposus of intervertebral discs \cite{solnica-krezel2012,balmer2016,debree2018,winkley2020,ghimire2021,stemple2004,stemple2005,choi2008,raj2008a,lawson2015}. Understanding notochord structure and function during gastrulation is essential to elucidate how perturbations to this machinery may lead to congenital vertebral defects. In this thesis dissertation, I demonstrated the importance of understanding the regulatory mechanisms driving gastrulation by studying the activity of enhancers and the contents of a cell during notogenesis. The ability to perform high-throughput experiments in the marine chordate \textit{Ciona intestinalis type A} or \textit{Ciona robusta} (\textit{Ciona}) contributed to the ability to screen the activity of thousands of enhancers to understand the contributions of transcription factor binding sites to function (Chapter \ref{chap:Diverse logics encode notochord enhancers}, Chapter \ref{chap:Proof of concept method to identify enhancers}). Likewise, \textit{Ciona} also granted us the unique ability to profile the transcriptomes of thousands of single cells in whole, gastrulating embryos to understand the contents of a cell across major tissues during cell type specification (Chapter \ref{chap:Ciona intestinalis gastrulation}).

    In Chapter \ref{chap:Diverse logics encode notochord enhancers}, we sought to understand the regulatory logic of notochord enhancers by taking advantage of the ability to perform high-throughput, massively-parallel reporter assays (MPRA) within \textit{Ciona}. Within the \textit{Ciona} genome, we identified 1,092 genomic regions, dubbed the ZEE library, containing a Zic binding site within 30 bp of an ETS binding site \cite{song2022}. Of the 90 ZEE elements, surprisingly, only nine drove notochord expression. One of the nine we identified, the \textit{Ciona} \textit{laminin alpha} enhancer, relied on grammatical constraints on Zic and ETS for functional activity. We also find similar clusters of Zic and ETS binding sites proximal to the mouse and human \textit{laminin alpha-1} gene with syntax similar to the \textit{Ciona laminin} enhancer \cite{song2022}. Within this chapter, we also highlight the importance of testing the sufficiency of TFBSs to investigate if we fully understand the regulatory logic of an enhancer. Through a randomization study, we reveal within the previously identified BraS enhancer that Zic and ETS binding sites are insufficient for notochord activity. Furthermore, we also find that FoxA and Bra sites are also necessary for notochord expression with BraS and that the combination of Zic, ETS, FoxA, and Bra binding sites may be a common logic regulating Bra expression \cite{song2022}. Our findings in Chapter \ref{chap:Diverse logics encode notochord enhancers} illustrate a common problem in mining genomic data for patterns, especially in mining genomes for functional enhancers based on the presence of TFBSs. We demonstrate that the presence of binding sites alone does not correlate to enhancer activity. To understand how enhancers regulate gene expression, we need to understand the number and types of TFBSs within an enhancer and the dependency between these sites, such as TFBS syntax and affinity \cite{jindal2021}. Overall, our findings illustrate the importance of enhancer grammar within developmental enhancers and hint at the conserved role of grammar and logic across chordates. 

    In Chapter \ref{chap:Proof of concept method to identify enhancers}, we continue upon the framework of Chapter \ref{chap:Diverse logics encode notochord enhancers} to understand the regulatory logic of notochord enhancers consisting of Zic, ETS, Bra, and FoxA binding sites. By virtue of a new \textit{Ciona} genome reference sequence release in 2019 \cite{satou2019}, we found a total of 4,344 genomic elements containing Zic and ETS binding sites with flexible constraints of the position of these sites within a 100 bp window. This library is otherwise known as the KYN library. After testing these KYN elements in an MPRA in whole \textit{Ciona} embryos, we found that only 15.4\% of these sites were active and dependent on Zic and ETS binding sites. Reviewing several candidate enhancers, we find they are proximal to multiple genes implicated in nervous system disorders and skeletal and bone-related disorders. Ultimately, further study of this enhancer library through imaging studies and TFBS ablation experiments is needed to ascertain the dependency of binding sites in determining functionality. To finish this chapter, we introduced a proof-of-concept Python package, \textbf{E}ntire \textbf{G}enome se\textbf{A}rches for \textbf{G}rammars of \textbf{E}nhancers (EnGAGE), that was developed to aid in efforts to understand the connection between genomic sequence and regulatory activity. This work represents the beginnings of a new paradigm to understand enhancers through elucidating how the organization of collections of TFBSs contributes to functional activity. 

    Finally, we move beyond genomic sequence to the contents of a cell in Chapter \ref{chap:Ciona intestinalis gastrulation}, where we develop a high-throughput, dense transcriptional atlas of \textit{Ciona} gastrulation. Just as the specific activity of enhancers is essential for successful development, the contents of a cell also dictate the formation of key cell types, such as the epidermis, endoderm, mesenchyme, heart, muscle, germ cells, notochord, and nervous system. In this study, we develop \textit{Ciona} embryos to the time points dictating gastrulation-the 4.5 hours post fertilization (hpf), 5.5 hpf, and 6.5 hpf stages representing the early gastrula or 110-cell stage, late gastrula, and early neurula stages of development \cite{satoh2014}. Once developed, the embryos are rapidly disassociated and processed for single-cell RNA sequencing. In total, we were able to profile 356,671 cells, allowing us to identify major tissues undergoing organogenesis and rare cell-type populations, such as the developing heart and germ cells. We also validate our map with fluorescent \textit{in situ} hybridization (FISH) imaging studies, visualizing canonical marker genes and novel marker genes within late gastrula \textit{Ciona} embryos. By providing a higher resolution single-cell atlas just spanning gastrulation, we anticipate that other groups can use the map generated in this study to identify conserved canonical and novel cell differentiation markers. Additionally, this resource will provide insight into the cell fate mechanisms governing organ formation during \textit{Ciona} gastrulation. 

    In the first three chapters of this work, I interrogate the regulatory players driving notogenesis from a genomic perspective through understanding the impact of binding site dependencies within enhancers (Chapter \ref{chap:Diverse logics encode notochord enhancers}, Chapter \ref{chap:Proof of concept method to identify enhancers}) and from a cellular perspective through cataloging the contents of major cell types by creating a transcriptional atlas of the developing \textit{Ciona} gastrula (Chapter \ref{chap:Ciona intestinalis gastrulation}). I also demonstrate that we cannot rely on bioinformatic identification of putative enhancers based on TFBS presence alone. In addition to elucidating the mechanisms driving notochord enhancer activity and the transcriptional landscape driving organogenesis, I also make the argument for studying enhancer grammar. To identify developmental enhancers accurately from genomic sequences, we need to understand the number and types of TFBSs present within a sequence and the dependency between these sites regarding syntax and binding affinity. Additionally, we need to understand the cellular context and transcriptional landscape in which these sequences are active. By studying these two elements in tandem, we can further understand how we transform from a single cell to a multicellular organism. 

    Finally, in the last chapter of this work, I demonstrate the importance of teaching bioinformatics and the strategies for managing an academically diverse classroom. The work presented in the first three chapters of this thesis dissertation required an understanding of programming, data visualization, molecular and developmental biology, and statistics to comment on enhancer grammar and cell type specification (Chapter \ref{chap:Diverse logics encode notochord enhancers}, Chapter \ref{chap:Proof of concept method to identify enhancers}, Chapter \ref{chap:Ciona intestinalis gastrulation}). As molecular biology steps into the world of big data to understand regulatory genomics, scientists must pick up bioinformatics skills. In Chapter \ref{chap:Bioinformatics education}, I share my experiences teaching bioinformatics curricula at the university level through the SIOB 242C, CMM 262, and BISB Bootcamp courses offered at the University of California, San Diego. One of the most considerable challenges I encountered in developing a comprehensive bioinformatics course was balancing the diverse academic backgrounds of students with creating experiences that do not isolate students based on their knowledge gaps in molecular biology and computer programming. To create a community environment in the classroom, I discuss my strategies in slowly introducing computational and biological concepts to reduce information overload and combat stereotype threat. Additionally, I discuss the benefits of introducing practical bioinformatics examples through student-paced, classroom-wide live programming sessions. Through this work, I want to enforce that learning bioinformatics can be made accessible through proper course design and empathetic instruction.

    %%%%%%%%%%%%%%%%%%%%%%%%%%%%%%%%%%%%%%%%%%%%%%%%%%%%%%%%%%%%%%%%%%%%%%%%%%%%
    \section{Limitations and Future Directions}
    %%%%%%%%%%%%%%%%%%%%%%%%%%%%%%%%%%%%%%%%%%%%%%%%%%%%%%%%%%%%%%%%%%%%%%%%%%%%
    Though this work represents essential steps forward in understanding the mechanisms behind enhancer grammar and cell type specification during gastrulation, there are still many avenues of study to continue down and important limitations to keep in mind.

    In Chapter \ref{chap:Diverse logics encode notochord enhancers} and Chapter \ref{chap:Proof of concept method to identify enhancers}, there are limitations regarding identifying functional enhancers and the ability to translate grammatical principles across species. For example, within Chapter \ref{chap:Diverse logics encode notochord enhancers}, we screened 90 ZEE elements for functionality; however, only 10\% were active in the notochord. Additionally, in Chapter \ref{chap:Proof of concept method to identify enhancers}, we screen for 4,344 KYN elements for functionality; however, only 15.4\% are active and reliant on Zic and ETS. While we anticipate that finding more notochord enhancers regulated by Zic, ETS, and possibly Bra and FoxA could better inform our understanding of the notochord enhancer grammar, finding these regions is highly limited. Combining assays of genomic regions with synthetic and random enhancer screens is thus needed to gain enough data to determine grammatical rules. With regards to our findings of possible conserved enhancer logic and grammar across chordates, we did not test the mouse \textit{laminin alpha-1} enhancer for activity in mouse for the study presented in Chapter \ref{chap:Diverse logics encode notochord enhancers}. We also did not functionally interrogate the importance of the 12 bp spacing within this enhancer in the context of \textit{Ciona} or mouse. Conducting these additional studies would deepen our understanding of the conservation of grammar across chordates. On the other hand, for the Zic, ETS, Bra, and FoxA logic found within \textit{Brachyury} enhancers in Chapter \ref{chap:Diverse logics encode notochord enhancers}, further manipulations of these TFBSs in the context of mouse and zebrafish \textit{Brachyury}/\textit{T}/\textit{TBXT} enhancers are required to determine if the conservation of logic is essential for the regulation of \textit{Brachyury}. Similar interrogations into the importance of binding sites in the active enhancers identified in Chapter \ref{chap:Proof of concept method to identify enhancers} are necessary to evaluate the components of the enhancer required for activity. 

    In Chapter \ref{chap:Ciona intestinalis gastrulation}, I present a high-resolution transcriptional atlas encompassing \textit{Ciona} gastrulation. Despite our success in identifying key cell types and sub-clusters representing their original cell-type lineages in \textit{Ciona} (e.g., A-line, B-line, a-line, and b-line), we did not evaluate the cell lineage specification pathways or pseudotime trajectories in the formation of these cell types. There is increasing interest in understanding the transitionary states involved in cell type specification. Within our dataset, one avenue for future study could be the initial formation of the notochord from mesenchymal tissue or the formation of initial neural subtypes from the A-line, a-line, and b-line cell lineages of the \textit{Ciona} embryo. Uncovering the markers delineating particular states in these trajectories, especially in a high-resolution single-cell atlas, may uncover additional essential marker genes involved in initial organ formation. In addition to studying cell type specification patterns through constructing pseudotime trajectories, another avenue for future work includes annotation of genes present in the \textit{Ciona} gastrula. Within our high-resolution single-cell atlas, we were able to identify not only canonical markers within cell types but also many novel markers lacking clear definitions on Aniseed besides sequence homology to vertebrate homologs. A straightforward avenue for future work is visualizing these novel markers through imaging experiments to validate their expression in the cell types identified in our single-cell atlas and define these genes for the larger \textit{Ciona} community. These studies would also confirm or deny the value of sequence homology in determining the true activity of novel markers.
    
\end{dissertationepilogue}