\chapter{Unraveling the sequence determinants of environment-induced changes in pancreatic beta-cell states}
\label{chap:Unraveling environment-induced changes in beta-cells}

%%%%%%%%%%%%%%%%%%%%%%%%%%%%%%%%%%%%%%%%%%%%%%%%%%%%%%%%%%%%%%%%%%%%%%%%%%%%%%%%
\section{Introduction}
%%%%%%%%%%%%%%%%%%%%%%%%%%%%%%%%%%%%%%%%%%%%%%%%%%%%%%%%%%%%%%%%%%%%%%%%%%%%%%%%

Pancreatic islets are XXX consisting of clusters of both exocrine and endocrine cells that regulate XXX {}. Among the endocrine cells, β cells play a central role by producing and secreting insulin in response to intrinsic {} and extrinsic signals {}. While glucose is the principal stimulus for β cell mediated insulin secretion {}, β-cell function is also modulated by an array of additional signals, including fatty acids {}, amino acids {}, incretin hormones {}, proinflammatory cytokines {}, glucocorticoids {}, and sex hormones {}. A few of these environmental cues have been shown to alter the broader transcriptional and epigenomic landscape of β cells {}, revealing widespread, time-dependent changes in histone modifications {}, chromatin accessibility {} and gene expression {}. These responses primarily reflect modulation of pre-existing cis-regulatory elements, which are frequently enriched for both transcription factor binding motifs relevant to the applied stimulus and glucose-associated genetic variants identified through genome-wide association studies {}.

Variants identified through genome-wide association studies (GWAS) of metabolic traits and related diseases, such as fasting glucose {} and type 2 diabetes {} respectively, are strongly enriched in cis-regulatory elements active in pancreatic islets {}. Within islets, this enrichment is pronounced in regulatory regions specific to β cells {}. For instance, common variants near MTNR1B, a locus contributing substantially to fasting glucose heritability, have been shown to alter β-cell enhancer activity {}. Similar effects have been reported at other loci, including those near transcription factors critical for β-cell identity and function, such as PDX1 and MAFA {}. More comprehensively elucidating the mechanisms by which noncoding variants impact β-cell regulatory programs remains a major challenge, particularly when a certain environmental context is necessary for the effect to be observed. For example, XXX.

Given the central role of islets in diabetes pathogenesis, the islet epigenome has been the focus of extensive investigation {}. Most of these data have been derived from bulk islet tissue, limiting the ability to resolve contributions from individual cell types. Recent advances in single-cell technologies have enabled high-resolution epigenomic profiling at the level of individual cells. Including multimodal platforms that simultaneously measure combinations of chromatin accessibility, gene expression, DNA methylation, and three-dimensional genome architecture {}. Our group and others have recently generated single-cell chromatin accessibility maps from healthy human islets, revealing regulatory elements active in β cells and uncovering previously unappreciated heterogeneity within the β-cell population {}. This includes variant XXX.

However, primary human islets are not amenable to genetic manipulation, which limits their utility for validating the functional consequences of these variants. Transformed human β-cell lines offer an alternative but face significant limitations, including poor single-cell growth and the absence of paracrine interactions with other endocrine cell types {}. Human pluripotent stem cell (hPSC)-derived islet organoids help overcome these barriers by recapitulating the major endocrine populations present in native islets {} while providing a genetically tractable system for targeted genome editing. CRISPR-based perturbations in hPSC-islet organoids have enabled both fine-scale dissection of individual regulatory elements and scalable functional screening. XXX

Deep learning models trained directly on genomic sequence have shown strong performance in predicting a range of regulatory phenotypes {}. These sequence-to-function models learn sequence features, including transcription factor binding site syntax and combinatorial motif logic, that are predictive of chromatin accessibility{}, gene expression{}, DNA methylation{}, and three-dimensional chromatin conformation{}. Because they operate on sequence alone, these models provide a natural framework for evaluating the regulatory impact of genetic variants. In particular, they can estimate the functional consequences of noncoding variants by comparing predicted outputs for reference and alternate alleles, offering a scalable approach to variant effect prediction. As such, they represent a valuable tool for decoding the regulatory genome and prioritizing candidate variants associated with disease-relevant traits.

Here we present a systematic framework for unraveling the sequence determinants of environment-induced changes in cell state. We first simultaneously measure the 

Our findings support the physiological relevance of in vitro stimulation paradigms for modeling disease-relevant regulatory dynamics.


%%%%%%%%%%%%%%%%%%%%%%%%%%%%%%%%%%%%%%%%%%%%%%%%%%%%%%%%%%%%%%%%%%%%%%%%%%%%%%%%
\section{Results}
%%%%%%%%%%%%%%%%%%%%%%%%%%%%%%%%%%%%%%%%%%%%%%%%%%%%%%%%%%%%%%%%%%%%%%%%%%%%%%%%

\subsection{Multiomic profiling of stem-cell derived pancreatic organoids}

\subsection{Cytokines induce strong transcriptomic response in SC-β cells}

\subsection{Cytokines induce widespread changes in chromatin accessibility}

\subsection{Sequence-to-function neural networks capture chromatin accessibility and predict variant effects}

\subsection{DNA methylation is relatively stable across short-term environmental perturbations}